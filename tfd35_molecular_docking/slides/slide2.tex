\frametitle{Epilepsy and Antiepileptic drugs}
\par As a definition, epilepsy is the name of a brain disorder characterized predominantly by recurrent and unpredictable interruptions of normal brain function \cite{fisher2005epileptic}. It affects people from both sexes and all ages. Approximately 50 million people worldwide have this neurological disorder \cite{world2006neurological}. It is characterized by repeated seizures, which are brief episodes of uncontrolled movement that may involve a part of the body (partial) or the entire body (generalized) and sometimes followed by loss of consciousness and control of bowel or bladder function. Mostly reasons of epileptic seizures are excessive and abnormal neuronal activity in the cortex of the brain \cite{fisher2005epileptic}. Seizure episodes result from excessive electrical discharges in a group of brain cells. Different parts of the brain can be the site of such discharges.
\newline
\par Antiepileptic drugs (AEDs) are used to prevent epileptic seizures. AEDs (such as Phenytoin, Mephenytoin, and Ethotoin) are a diverse group of pharmacological agents used in the treatment of epileptic seizures. AEDs suppress the excessive rapid discharge of neurons during seizures. AEDs also prevent the spread of the seizure within the brain. Seizures can be controlled up to 70\% of people living with epilepsy could become seizure free with use of antiseizure medications \cite{sander1993some}.